\section{Uvod}
Grafi so ena izmed najbolj uporabnih struktur v matematiki. Služijo kot orodje pri reševanju problemov v drugih vejah matematike, kot pripomoček za modeliranje realnih situacij ali pa jih preučujemo kot samostojen matematični koncept. Pogosto potrebujemo grafe, ki imajo visoko količino povezav.

Poglejmo konkreten primer iz vsakdanjega življenja. Vozlišča v grafu naj predstavljajo mesta, povezave med njimi pa ceste, ki vodijo med mesti. Želimo si, da bi bilo mogoče iz vsakega mesta potovati do katerega koli drugega mesta. To drži natanko tedaj, ko je graf povezan. Lahko pa bi se zgodilo, da neka cesta postane neprevozna, na primer zaradi prometne nesreče ali naravne katastrofe. To predstavimo tako, da odstranimo povezavo v grafu. V hujših primerih bi lahko prišlo do uničenja več cest. Kljub temu želimo, da naš graf ostane povezan ter tako lahko še vedno potujemo med mesti in je naša infrastruktura bolj odporna proti motnjam. Najbolj robustno bi bilo omrežje, v katerem bi bilo vsako mesto povezano z vsemi ostalimi, vendar bi bilo to predrago, zato si želimo, da je število povezav majhno, še vedno pa jih potrebujemo odstraniti veliko, preden graf postane nepovezan.
Podoben primer najdemo v električnih omrežjih, kjer vozlišča predstavljala transformatorske postaje, povezave pa daljnovode med njimi. Lahko bi pa za vozlišča uporabljali usmerjevalnike ter za povezave internetne kable. V vseh primerih želimo dobiti čim bolj odporno omrežje za čim manjšo ceno infrastrukture. Ramanujanovi grafi so grafi, ki so na nekakšen način optimalni glede na opisano metriko.
