\section{Uvod}
Grafi so ena izmed najbolj uporabnih struktur v matematiki. Uporabljamo jih kot pomoč pri reševanju problemov drugih vej matematike, kot orodje pri reševanju težav iz resničnega sveta ali pa jih preučujemo neodvisno od ostalih okoliščin. Pogosto potrebujemo grafe, ki imajo visoko količino povezav.

Poglejmo primer iz vsakdanjosti. Vozlišča v grafu naj predstavljajo mesta, povezave med njimi pa ceste, ki vodijo med mesti. Želimo si, da bi bilo mogoče iz vsakega mesta potovati do katerega koli drugega mesta. To velja natanko takrat, ko je graf povezan. Lahko pa bi se zgodilo, da neka cesta postane neprevozna, na primer zaradi prometne nesreče ali naravne katastrofe. To predstavimo tako, da odstranimo povezavo v grafu. V hujših primerih bi se lahko zgodilo tudi da se uniči več cest. Kljub temu želimo, da naš graf ostane povezan ter tako lahko še vedno potujemo med mesti in je naša infrastruktura bolj odporna proti motnjam. Najbolj odporni bi bili seveda, če bi bilo vsako mesto povezano z vsemi ostalimi, vendar to bi bilo predrago, zato si želimo, da je število povezav majhno, še vedno pa jih potrebujemo odstraniti veliko preden graf postane nepovezan.
Podoben primer bi lahko dobili, če bi vozlišča predstavljala transformatorske postaje, povezave pa žice med njimi. Lahko bi pa za vozlišča uporabljali usmerjevalnike ter za povezave internetne kable. V vseh primerih želimo dobiti čim bolj odporno omrežje za čim manjšo ceno infrastrukture. Ramanujanovi grafi so grafi, ki so na nekakšen način optimalni glede na opisano metriko.
