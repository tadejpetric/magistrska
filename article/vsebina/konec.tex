\section{Zaključek}
Ogledali smo si Ramanujanove grafe, kar nas je peljalo skozi mnogo disciplin matematike. Začeli smo z uporabno motivacijo iz resničnega življenja, motivirali povezanost s pojmi iz diskretne matematike, definicija Ramanujanovih grafov pa je naraven primer algebraične teorije grafov. Ko smo grafe iskali računsko, smo uporabili računalništvo, pri eksplicitni konstrukcijah pa smo uporabljali elementarno verjetnost, abstraktno algebro, analizo in teorijo števil. Zaključili pa smo z izrekom, ki poveže Ramanujanove grafe z eno izmed najpomembnejših domnev v kompleksni analizi in teoriji števil (in matematiki nasploh). Kljub temu, da so veje matematike pogosto povezane med seboj, je težko poiskati kakšno temo, ki bi se hitreje dotaknila vseh teh različnih vej matematike -- že sama konstrukcija, kar je običajno prva stvar, ki si jo želimo ogledati, ko definiramo nov koncept, nas popelje do vseh različnih vej matematike ter jih uporablja v ključnih korakih. Raziskave o Ramanujanovih grafih pa so še vedno zelo aktivne. Iščejo se boljše metode konstrukcije \cite{marcus2015interlacingfamiliesivbipartite}, praktične uporabe (na primer v kriptografiji \cite{10.1007/978-3-030-19478-9_1}), njihove povezave z naključnimi grafi \cite{huang2024ramanujanpropertyedgeuniversality} in še marsikaj drugega. 