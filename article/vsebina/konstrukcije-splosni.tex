\section{Konstrukcija splošnih Ramanujanovih grafov}
Ogledali si bomo še zadnjo konstrukcijo Ramanujanovih grafov, ki pa je pravzaprav bila kronološko prva. Ta nam bo poiskala neskončno družino ramanujanovih grafov stopnje \(p+1\) za poljubno praštevilo \(p\). Za razliko od prejšne konstrukcije pa ne bomo dobili samih dvodelnih grafov.

Glavni razlog, da si bomo ogledali to konstrukcijo pa je, da je zanimiva in uporablja povsem drugačne pristope kot prej. Skica konstrukcije je sledeča: poiskali bomo družino matričnih grup, ki ima \(p+1\) generatorjev. Za vsako grupo lahko poiščemo njen Caylejev graf, ki ima enako stopnjo regularnosti kot je število generatorjev grupe, torej dobimo \((p+1)\)-regularen graf. Za te grafe pa se bo izkazalo, da so Ramanujanovi.

Vsi rezultati, če ni citirano drugače, so iz članka \cite{lps-ramanujan}. % TODO: nekako polepšaj ta stavek

\subsection{Teorija števil}
Začnemo z nekaj osnovnimi definicijami in izreki iz teorije števil, ki jih bomo potrebovali v konstrukciji grafov.

Prva definicija generalizira pojem popolnega kvadrata na končna polja. V celih številih so popolni kvadrati števila oblike \(k^2\), kjer je \(k\) celo število. Če pa to gledamo v kolobarju \(\mathbb Z/n\mathbb Z\) dobimo enako definicijo, samo modulo \(n\).
\begin{definicija}[Kvadratni ostanek]
    Število \(a\in \mathbb Z\) je kvadratni ostanek modulo \(n\), če obstaja takšno takšno število \(k\in \mathbb Z\), da velja \(a\equiv k^2 \pmod n\). 
\end{definicija}

Sedaj lahko definiramo Legendrejev simbol.
\begin{definicija}[Legendrejev simbol]
    Za število \(a\in \mathbb Z\) in \(p>2\) praštevilo. Legendrejev simbol je definiran kot
    \begin{align*}
        \legendre{a}{p} = 
        \begin{cases}
            1 &a \neq 0 \text{ in } a \text{ je kvadratni ostanek modulo } p\\
            0 &a \equiv 0 \pmod p \\
            -1 & a \text{ ni kvadratni ostanek modulo } p
        \end{cases}
    \end{align*} 
\end{definicija}

\begin{primer}
    Oglejmo si kvadratne ostanke modulo 5.
    Velja
    \begin{align*}
        0 \equiv 0^2 \pmod 5\\
        1^2 \equiv 4^2 \equiv 1 \pmod 5\\
        2^2 \equiv 3^2 \equiv 4 \pmod 5.
    \end{align*} 
    Če vzamemo število večje od \(5\) ga lahko zapišemo kot \(k\cdot 5 + r\) (za \(0\leq r <5\)), kvadrat tega izraza pa je enak \(k^2\cdot 5^2 + 2kr\cdot 5 + r^2\). Ko ga gledamo modulo \(5\) pa ostane le sumand \(r^2\) in ne moremo zapisati novih opcij. Torej so kvadratni ostanki modulo \(5\) števila oblike \(5\cdot k\), \(5\cdot k + 1\) in \(5\cdot k + 4\).

    Tako lahko napišemo tudi Legendrejeve simbole. Recimo
    \begin{align*}
        \legendre{5k}{7} = 0 \\
        \legendre{5k + 1}{7} = \legendre{5k + 4}{7} = 1\\
        \legendre{5k + 2}{7} = \legendre{5k + 3}{7} = -1.
    \end{align*}
\end{primer}

Izbiro grupe bomo pogojevali glede na Legendrejev simbol. Če bomo želeli narediti \(p+1\)-regularen graf bomo izbrali še dodatno praštevilo \(q\). Nato bomo, v primeru \(\legendre{p}{q}=1\) izbrali grupo \(\PSL(2, \modZ{q})\), če je pa \(\legendre{p}{q}=-1\) pa grupo \(\PGL(2, \modZ{q})\). Te grupi si bomo kasneje ogledali bolj podrobno, za nas pa bo pomembno, da imata obe grupi \(p+1\) generatorjev. Da bi to dokazali, potrebujemo še en izrek iz teorije števil.

\begin{izrek}[Jacobijev izrek o štirih kvadratih]
    
\end{izrek}

\subsection{Caylejevi grafi}
