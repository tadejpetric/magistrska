\section{Zeta funkcije grafov in Riemannova hipoteza}
Večina praktičnih uporab Ramanujanovih grafov prazvaprav ne zahteva, da so grafi točno Ramanujanovi. Običajno imamo nek postopek za katerega velja, da če je naš graf bolje povezan (oziroma, da je druga največja lastna vrednost čim večja), bomo dobili boljše rezultate. Velja sicer, da so Ramanujanovi grafi optimalni na ta kriterij, torej bi za njih dobili najboljše rezultate, niso pa strogo potrebni. Želimo pa si ogledati kakšno uporabo ali lastnost Ramanujanovih grafov, kjer bi bilo zares ključno, da so grafi Ramanujanovi. Zato si bomo ogledali eno izmed najbolj pomembnih domnev v matematiki, Riemannovo hipotezo. Ta opisuje obnašanje ničel določene kompleksne funkcije, vendar se izkaže, da obstaja analog te funkcije na grafih; vsakemu regularnemu grafu pripada neka kompleksna funkcija. Izrek bomo natančno formulirali kasneje, vendar pa dokaza za navadno Riemannovo hipotezo še ne poznamo. Zanimivo pa je, da je za različico z grafi dokaz že znan: izkaže se, da funkcija, ki pripada grafu, zadošča Riemannovi hipotezi natanko takrat, ko je graf Ramanujanov\cite{portablesnowbird}.

\subsection{Riemannova Zeta funkcija}
Začeli bomo z navadno Riemannovo Zeta funkcijo\cite{freitag1}. Enostavno jo je definirati kot analitično funkcijo na \(\Re(z)>1\) z
\begin{align*}
    \zeta(z) &= \sum_{n=1}^\infty \frac{1}{n^z} \\ 
    &= \prod_{p \text{ praštevilo}} \left(1-p^{-z}\right)^{-1}.
\end{align*}
Funkcijo \((1-z)\zeta(z)\) lahko analitično razširimo na celotno kompleksno ravnino, torej lahko tudi \(\zeta\) razširimo v meromorfno funkcijo na \(\C\) z polom v \(z=1\). Že iz alternativnega zapisa funkcije vidimo, da je tesno povezana s praštevili, Riemann pa je razširjeno različico uporabil, da je dokazal praštevilski izrek, ki pravi, da število praštevil manjših od števila \(n\) asimptotsko raste z \(\frac{x}{\log{x}}\). Ničle \(\zeta\) funkcije pa nam o tem izreku lahko povejo še več, z njimi lahko opišemo napako izraza iz praštevilskega izreka in s tem bolje razumemo porazdelitev praštevil.

Ničle Riemannove Zeta funkcije delimo na trivialne in na netrivialne. Najprej si oglejmo trivialne ničle ter območja, kjer je enostavno videti, da funkcija nima ničel. Nerazširjena Zeta funkcija (torej na vrednostih \(\Re z > 1\)) nima ničel. Praštevilski izrek je ekvivalenten temu, da na osi \(\Re z = 1\) ni ničel\cite{harolddiamond}. Za negativno polravnino (\(\Re z \leq 0\)) je enostavno poiskati ničle, tu dobimo ``trivialne ničle'', ki se nahajajo na \(-2k\) za \(k\in\Z^+\). Drugih ničel na tej polravnini ni.

Ostane nam samo še množica \(\{z\in \C\mid 0 < \Re z < 1\}\), ki jo imenujemo kritični trak. Tu se nahajajo netrivialne ničle, točne distribucije pa ne poznamo. Znotraj traka se nahaja kritična premica \(\{z\in \C \mid \Re z = 1/2\}\). Znano je, da se na tej premici nahaja več kot 40\% netrivialnih ničel\cite{Pratt2019}. Računsko smo izračunali že vsaj 100 miljard ničel Zeta funkcije na kritičnem traku in vse so na kritični premici\cite{racunskoniclezeta}. To nas pripelje do Riemannove hipoteze.

\begin{izrek}[Riemannova hipoteza]
    Vse netrivialne ničle Zeta funkcije se nahajajo na kritični premici.
\end{izrek}

\subsection{Ihara Zeta funkcija}
Sedaj bomo definirali Zeta funkcijo, ki pripada grafu. Definicija te funkcije bo analogna definiciji klasične Zeta funkcije, še bolj pa je sorodna Selbergovi Zeta funkciji, katere si ne bomo ogledali\cite{sunada-zetagrafov}.

Z \(s=(v_0, \ldots, v_{l})\) označimo sklenjen okrajšan sprehod v grafu. To je sprehod, ki ne obišče dveh vozlišč zaporedoma (torej \(v_i \neq v_{i+2}\), kot v definiciji \ref{okrajsana-beseda}), začne in konča pa se v istem vozlišču (torej \(v_0 = v_l\)). Z \(L(s)=l\) označimo dolžino takega sprehoda. Bolj kot zaporedje vozlišč nas zanima oblika poti v grafu neodvisno na začetno točko. Če dodamo še zahtevo \(v_1 \neq v_{l-1}\) lahko obravnavamo vse take sprehode kot ekvivalentne (torej dva sprehoda sta ekvivalentna, če se zaporedje vozlišč razlikuje za ciklično permutacijo). To nam definira pojem zaprte geodezijke v grafu. 

\begin{definicija}[Zaprta geodezijka grafa]
    Zaprta geodezijka grafa je ekvivalenčni razred sprehodov \(s\) dolžine \(l\) za katere velja
    \begin{align*}
        v_0 = v_l \\
        v_i \neq v_{i+2 \pmod l},
    \end{align*}
    dva sprehoda \(s=(v_0, \ldots, v_{l})\) in \(s'=(v_0', \ldots, v_{l}')\) pa sta si ekvivalentna, če obstaja ciklična permutacija \(\pi\) na \(l-1\), da \(v_{\pi(i)} = v_i'\). 
\end{definicija}

Če geodezijke ne moremo dobiti tako, da večkrat ponovimo krajšo pot krajšo pot, je ta geodezijka nerazcepna. Na primer, v 3-ciklu nam pot \((v0, v1, v2, v0)\) definira zaprto nerazcepno geodezijko, pot \((v0, v1, v2, v0, v1, v2, v0)\) pa zaprto razcepno geodezijko. Namesto praštevil v definiciji zeta funkcije bomo iterirali po nerazcepnih geodezijkah.

\begin{definicija}[Ihara zeta funkcija]
    Ihara Zeta funkcija grafa \(G\) je analitično nadaljevanje funkcije
    \begin{align*}
        \zeta_G(z) = \prod_{p}\left(1-z^{L(p)}\right)^{-1},
    \end{align*}
    kjer produkt iterira po vseh nerazcepnih geodezijkah grafa \(G\).
\end{definicija}
 
Z \(\alpha_G\) označimo konvergenčni radij vrste. Za \(d\)-regularne grafe je ta enak kar \(\alpha_G = 1/(d-1)\), Ihara Zeta funkcija pa je kar racionalna funkcija. V primeru regularnih grafov lahko funkcijo opišemo še bolje: izkaže se namreč, da je recipročna funkcija \(\zeta_G^{-1}\) cela funkcija, torej \(\zeta_G\) nima ničel. Od tu naprej bodo vsi grafi regularni.

% https://en.wikipedia.org/wiki/Ihara_zeta_function#cite_note-2
% Portable snowbird
\begin{izrek}[Ihara Zeta je racionalna]
    Za \(d\)-regularen graf \(G\) in \(r(G)-1 = (d-2) n / 2\) je Ihara Zeta funkcija enaka
    \begin{align*}
        \zeta_G(z) = \left((1-z^2)^{r(G)-1}\det(I-Az+(d-1)z^2I)\right)^{-1}.
    \end{align*}
\end{izrek}

Ker funkcija nima ničel, si bomo ogledali pole Ihara Zeta funkcije (oziroma ničle funkcije \(\zeta_G^{-1}\)). Če ima graf \(m\) povezav je teh polov \(2m\). Podobno kot pri ničlah Zeta funkcije, se tudi poli Ihara Zeta funkcije nahajajo le na določenem delu kompleksne ravnine. Če uporabimo substitucijo \(w = (d-1)^{-z}\), dobimo vrsto, ki nima polov za \(\Re(w)>1\) ter ima pol prvega reda v \(w=1\). Če to navežemo na lastne vrednosti grafov, lastnost, da ni polov z \(\Re(w)>1\) ustreza lastnosti regularnih grafov, da nimajo lastnih vrednosti večjih od \(d\). Imajo pa eno lastno vrednost enako \(d\), kar ustreza polu prvega reda v \(w=1\). Kritični trak je tako, podobno kot pri klasični Riemannovi Zeta funkciji, trak \(0<\Re(w)<1\). Riemannova hipoteza za Ihara Zeta funkcijo pa je analogna klasični Riemannovi hipotezi - da imajo vse ničle, ki se nahajajo na kritičnemu traku, realno komponento enako \(1/2\).