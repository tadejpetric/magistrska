\subsection{Pisanje algoritmov}
Za pisanje algoritmov sta na voljo okolji \texttt{algorithm} in
\texttt{algorithmic} iz paketov \texttt{algorithm} in \texttt{algorithmix}, ki
sodelujeta podobno kot \texttt{table} in \texttt{tabular}. Algoritmi plavajo
med tekstom, enako kot slike in tabele, nanje se lahko tudi sklicujemo, kot
prikazano v izvorni kodi in v algoritmu~\ref{alg:metoda}. Sklicujemo se lahko
tudi na pomembne vrstice, npr.\ na vrstico~\ref{alg:pomembna-vrstica}, ki
predstavlja glavni del algoritma. Za primer pisanja algoritma se posvetujte s
primerom v tem dokumentu, za bolj napredne primere uporabe, kot na primer
razbijanje algoritma na več kosov, pa z (precej razumljivo) uradno
dokumentacijo\footnote{\url{http://tug.ctan.org/macros/latex/contrib/algorithmicx/algorithmicx.pdf}}.
Če želite vključiti izvorno kodo nekega programa, priporočamo paket
\texttt{minted}\footnote{\url{https://github.com/gpoore/minted}}.


\begin{algorithm}[ht]
  \caption{Opis, ki ima enako funkcionalnost kot opis pod sliko.}
  \label{alg:metoda}
  \raggedright
  \textbf{Vhod:} Števili $n, m \in \N, n > m$. \\
  \textbf{Izhod:} Decimalno število $x$, ki aproksimira rešitev enačbe $n x = m$.
  \begin{algorithmic}[1]
    \Function{reši}{$n$, $m$} \Comment{Vsi vhodni parametri morajo biti opisani.}
    \State $a \gets [\,]$ \Comment{Spremenljivka $a$ naj postane prazna kopica.}
    \For{$i$}{$1$}{$n$}
      \If{$i \operatorname{mod} 7 = 5$}
        \State \Call{heapop}{$a$}
      \ElsIf{$i < 5$}
      \State \Call{heappush}{$a, \frac{i+12}{7} + \pi$} \Comment{Lahko uporabljamo matematiko.}
      \Else
        \State \Call{heappush}{$a, i$}
      \EndIf
    \EndFor
    \Statex  \Comment{Prazna vrstica}
    \State $x \gets 0$  \Comment{To je primer komentarja.}
    \ForEach{e}{a}
      \State $x \gets 1 + \sqrt[e]{x}$
    \EndFor
    \While{$|x| > \varepsilon$}
      \State $x \gets x / 2$
    \EndWhile
    \State $x \gets m / n$ \label{alg:pomembna-vrstica}
    \State \Return $x$  \Comment{Vsi izhodni parametri morajo biti opisani nad algoritmom.}
    \EndFunction
  \end{algorithmic}
\end{algorithm}