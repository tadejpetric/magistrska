% naložite dodatne pakete, ki jih potrebujete
\usepackage{algpseudocode}  % za psevdokodo
\usepackage{algorithm}      % za algoritme
\usepackage{mathtools}
\usepackage{tikz}
\usetikzlibrary{positioning}
\usepackage{pgf}
\usepackage{pgfplots}
\floatname{algorithm}{Algoritem}
\renewcommand{\listalgorithmname}{Kazalo algoritmov}
\usepackage{layouts}

% deklarirajte vse matematične operatorje, da jih bo LaTeX pravilno stavil
% \DeclareMathOperator{\conv}{conv}
% na razpolago so naslednja matematična okolja, ki jih kličemo s parom
% \begin{imeokolja}[morebitni komentar v oklepaju] ... \end{imeokolja}
%
% definicija, opomba, primer, zgled, lema, trditev, izrek, posledica, dokaz

% za številske množice uporabite naslednje simbole
\newcommand{\R}{\mathbb R}
\newcommand{\N}{\mathbb N}
\newcommand{\Z}{\mathbb Z}
% Lahko se zgodi, da je ukaz \C definiral že paket hyperref,
% zato dobite napako: Command \C already defined.
% V tem primeru namesto ukaza \newcommand uporabite \renewcommand
\newcommand{\C}{\mathbb C}
\newcommand{\Q}{\mathbb Q}
\newcommand{\bH}{\mathbb H}

% https://tex.stackexchange.com/questions/43008/absolute-value-symbols
\DeclarePairedDelimiter\abs{\lvert}{\rvert}%
\DeclarePairedDelimiter\norm{\lVert}{\rVert}%
\DeclarePairedDelimiter{\ceil}{\lceil}{\rceil}
\DeclarePairedDelimiter{\floor}{\lfloor}{\rfloor}

% Swap the definition of \abs* and \norm*, so that \abs
% and \norm resizes the size of the brackets, and the 
% starred version does not.
\makeatletter
\let\oldabs\abs
\def\abs{\@ifstar{\oldabs}{\oldabs*}}
%
\let\oldnorm\norm
\def\norm{\@ifstar{\oldnorm}{\oldnorm*}}
%
\let\oldceil\ceil
\def\ceil{\@ifstar{\oldceil}{\oldceil*}}
%
\let\oldfloor\floor
\def\floor{\@ifstar{\oldfloor}{\oldfloor*}}
\makeatother


\DeclareMathOperator{\sgn}{sgn}

\def\mathdefault#1{#1}
\everymath=\expandafter{\the\everymath\displaystyle}
\IfFileExists{scrextend.sty}{
\usepackage[fontsize=10.000000pt]{scrextend}
}{
\renewcommand{\normalsize}{\fontsize{10.000000}{12.000000}\selectfont}
\normalsize
}

\makeatletter\@ifpackageloaded{underscore}{}{\usepackage[strings]{underscore}}\makeatother

\DeclareMathOperator{\GL}{GL}
\DeclareMathOperator{\SL}{SL}
\DeclareMathOperator{\PSL}{PSL}
\DeclareMathOperator{\PGL}{PGL}
\DeclareMathOperator{\Tr}{Tr}

\newcommand{\modZ}[1]{\Z/#1\Z}

\newcommand{\legendre}[2]{\left(\frac{#1}{#2}\right)}